% vim: set tw=78 aw:
\documentclass{beamer}

\usepackage[utf8x]{inputenc} % diacritice
\usepackage[romanian]{babel}
\usepackage{color}			 % highlight
\usepackage{alltt}			 % highlight
\usepackage{code/highlight}	 % highlight
\usepackage{hyperref}        % folositi \url{http://...}
% sau \href{http://...}{Nume Link}
\mode<presentation>
{ \usetheme{Rochester} }		% TODO: settle this

% Titlul nu foloseşte Unicode pentru că e o problemă căreia nu i-am dat de
% cap.
\title[Systemtap]{Systemtap}
\subtitle{Profiling Tech Talks}
\institute{ROSEdu}
\author{Lucian Adrian Grijincu \\ \texttt{lucian.grijincu@rosedu.org}}


\newcommand{\systemtap}{$\texttt{systemtap}$ } 
\begin{document}

% Slide-urile cu mai multe părţi sunt marcate cu textul (cont.)
\setbeamertemplate{frametitle continuation}[from second]
% Arătăm numărul frame-ului
\setbeamertemplate{footline}[frame number]

\frame{\titlepage}

\frame{\tableofcontents}

% NB: Secţiunile nu sunt marcate vizual, ci doar apar în cuprins.
\section{Ce e systemtap?}

% Pentru reamintirea periodică a cuprinsului şi unde ne aflăm:
\frame{\tableofcontents[currentsection]}


\begin{frame}{Cum lucrăm fără \systemtap?}
  \begin{beamerboxesrounded}[lower=block body,shadow=true]{un program cu bug-uri (câte ați găsit?)}
    \small \input{code/printf-debugging-without}
  \end{beamerboxesrounded}
\end{frame}


\begin{frame}{Cum lucrăm fără \systemtap?}
  \begin{beamerboxesrounded}[lower=block body,shadow=true]{băgăm niște printfuri}
    \small \input{code/printf-debugging-with-1}
  \end{beamerboxesrounded}
\end{frame}

\begin{frame}{Cum lucrăm fără \systemtap?}
  \begin{beamerboxesrounded}[lower=block body,shadow=true]{corectăm și mai băgăm niște printfuri}
    \small \input{code/printf-debugging-with-2}
  \end{beamerboxesrounded}
\end{frame}

\begin{frame}{Cum lucrăm fără \systemtap?}
  \begin{beamerboxesrounded}[lower=block body,shadow=true]{și tot așa ...}
    \small \input{code/printf-debugging-with-3}
  \end{beamerboxesrounded}
\end{frame}

\begin{frame}{Cum lucrăm fără \systemtap?}
  \begin{beamerboxesrounded}[lower=block body,shadow=true]{la final, curățăm programul}
    \small \input{code/printf-debugging-done}
  \end{beamerboxesrounded}
\end{frame}

\begin{frame}{Cum lucrăm fără \systemtap?}
  \begin{itemize}
    \item repetitiv
    \item pentru fiecare \texttt{printf} trebuie recompilat proiectul
      (timpii de compilare pot fi foarte mari)
    \item poate introduce alte probleme
    \item nu prea este la îndemâna administratorilor de sistem
    \item după ce se rezolva problema, sunt șterse, și dacă apare mai
      târziu o problemă în același loc, sunt rescrise
    \item ...
  \end{itemize}
\end{frame}


\begin{frame}{Cu ce ajută \systemtap?}
  \begin{itemize}
    \item permite \textit{administratorilor} și
      \textit{programatorilor} să scrie scripturi simple cu care să
      monitorizeze activitatea internă a kernelului sau a aplicațiilor
    \item datele pot fi
      \begin{itemize}
      \item extrase
      \item filtrate
      \item sumarizate
      \end{itemize}
    \item totul în siguranță, fără a introduce probleme noi
    \item \small există și posibilitatea modificării rezultatelor
  \end{itemize}
\end{frame}









\section{Instalare (Ubuntu)}

\begin{frame}{Instalare (Ubuntu)}
  \begin{itemize}
    \item \systemtap are nevoie de:
      \begin{itemize} 
      \item suport în kernel
        \begin{beamerboxesrounded}[lower=block body,shadow=true]
          \small \input {code/install--make-menuconfig}
        \end{beamerboxesrounded}
      \item simboluri de debugging
      \end{itemize}
    \item în Ubuntu, suportul este compilat în kernel
    \item simbolurile de debugging pot fi descărcate
      \begin{itemize}
      \item http://ddebs.ubuntu.com karmic main
      \item wget http://ddebs.ubuntu.com/pool/main/l/linux/linux-image-debug-2.6.31-16-generic\_2.6.31-16.53\_i386.ddeb
      \end{itemize}
  \end{itemize}
\end{frame}


\begin{frame}{Testarea instalării}
  \input {code/install--test-script}
\end{frame}


\section{Demo}

\begin{frame}{system-wide \texttt{strace}}
  \small \input{code/strace}
\end{frame}

\begin{frame}{nr apeluri de sistem/pid}
  \small \input{code/syscalls_by_pid}
\end{frame}

\begin{frame}{nr apeluri de sistem/proces}
  \small \input{code/syscalls_by_proc}
\end{frame}

\section{Explicații}
\begin{frame}{Cum funcționează \systemtap?}
  \begin{itemize}
  \item traduce scripturile în C (CPL)
  \item compilează din ele un modul de kernel (CPL)
  \item introduce modulul în kernel (PSO)
  \end {itemize}
\end{frame}

    
\begin{frame}{Cum funcționează \systemtap?}
  \begin{itemize}
  \item la introducerea modulului: \\ \small activează evenimentele
    monitorizate (punctele de colectare a evenimentelor trebuie să fie
    deja definite)
  \item la declanșarea unui eveniment monitorizat: \\ \small se
    execută rutinele specificate pentru evenimentul monitorizat
  \item la oprirea programului \texttt{stap}: \\ \small dezactivează
    evenimentele și se descarcă modulul
  \end {itemize}
\end{frame}

\section{User space/kernel space}

\begin{frame}{Cui îi pasă de kernel, vrem să inspectăm temele scrise în C}
  \begin{itemize}
    \item se definesc probele ce se inspectează:
    \item \input {code/userspace-probe}
    \item din fișierul cu probele se obțin un fișier antet și un
      fișier obiect cu tot ce e nevoie pentru a folosi noile probe
    \item \input {code/userspace-makefile}
  \item nu e nevoie să modificați codul sursă (în afara unui \\ \#include ``probes.h'')
  \end{itemize}
\end{frame}

\begin{frame}{Cui îi pasă de C, vrem să inspectăm temele scrise în Python}
  \begin{itemize}
    \item recent s-a introdus suport pentru inspectarea codului Python
    \item expect it soon in a distro release near you
  \end{itemize}
\end{frame}


\begin{frame}{Î \& R}
  \begin{centering}
    \Huge ?    \par
  \end{centering}
\end{frame}

\section{Moar demos!}
\begin{frame}
Toate exemplele sunt furate de pe pagina proiectului: http://sourceware.org/systemtap/examples/index.html
\end{frame}

\end{document}
