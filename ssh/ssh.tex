% vim: set tw=78 aw:
\documentclass{beamer}

\usepackage[utf8x]{inputenc} % diacritice
\usepackage[romanian]{babel}
\usepackage{hyperref}        % folositi \url{http://...}
% sau \href{http://...}{Nume Link}
\mode<presentation>
{ \usetheme{Rochester} }		% TODO: settle this

% Titlul nu foloseşte Unicode pentru că e o problemă căreia nu i-am dat de
% cap.
\title[Secure Shell]{Secure Shell}
\subtitle{RTT - Cursul 2}
\institute{ROSEdu}
\author{Răzvan Deaconescu, Lucian Cojocar}

\begin{document}

% Slide-urile cu mai multe părţi sunt marcate cu textul (cont.)
\setbeamertemplate{frametitle continuation}[from second]
% Arătăm numărul frame-ului
\setbeamertemplate{footline}[frame number]

\frame{\titlepage}

\frame{\tableofcontents}

% NB: Secţiunile nu sunt marcate vizual, ci doar apar în cuprins.
\section{No\c{t}iuni generale}

% Pentru reamintirea periodică a cuprinsului şi unde ne aflăm:
\frame{\tableofcontents[currentsection]}

% Titlul unui frame se specifică fie în acolade, imediat după \begin{frame},
% fie folosind \frametitle
\begin{frame}{SSH}
  \begin{itemize}
    \item \textbf{protocol} de rețea (nivel aplicație)
    \item canal securizat de comunicație
    \item comandă de la distanță, transfer de fișiere, tunelare
    \item client (ssh), server (sshd, portul 22)
    \item servere pe Unix-uri, Mac OS X, Windows (cygwin)
    \item \href{http://www.openssh.com/}{OpenSSH} este practic singura
implementare de SSH
  \end{itemize}
\end{frame}

\begin{frame}{Criptare}
  \begin{itemize}
    \item două tipuri
      \begin{itemize}
        \item chei simetrice - rapidă
        \item chei asimetrice - lentă dar potențial mai sigură
      \end{itemize}
    \item algoritmi de criptare
       \begin{itemize}
         \item alg(mesaj) $\rightarrow$ mesaj\_criptat
         \item AES (Advanced Encryption Standard), Blowfish
         \item RSA (Rivest, Shamir, Adleman) , DSA (Digital Signature Algorithm)
       \end{itemize}
  \end{itemize}
\end{frame}

\begin{frame}{Chei publice/private}
  \begin{itemize}
    \item legătură matematică (ex: RSA se bazează pe teorema lui Euller)
    \item generate în același timp
    \item pub(priv(M)) = M (semnătură digitală)
    \item priv(pub(M)) = M (criptare)
    \item autentificare (certificate)
    \item criptare folosită în conjuncție cu chei simetrice
  \end{itemize}
\end{frame}

\section{Comenzi de bază}
\frame{\tableofcontents[currentsection]}

\begin{frame}{Fişiere utilizate (client)}
	\begin{itemize}
		\item ~/.ssh/id\_rsa cheia privată implicită
		\item ~/.ssh/id\_rsa.pub cheia publică implicită
		\item /etc/ssh/ssh\_config ~/.ssh/config 
		\item ~/.ssh/authorized\_keys chei publice pentru care nu se mai cere
			parola
		\item ~/.ssh/known\_hosts chei publice ale serverlor la care ne-am
			conectat
		\item ~/.shosts autentificare în funcție de host
	\end{itemize}
\end{frame}

\begin{frame}{Fişiere utilizate (server)}
	\begin{itemize}
		\item /etc/ssh/ssh\_host\_rsa\_key cheia privată a serverului
		\item /etc/ssh/ssh\_host\_rsa\_key.pub cheia publică a serverului
		\item /etc/ssh/sshd\_config
		\item /etc/ssh/ssh\_rc	comenzi executate la conectare
	\end{itemize}
\end{frame}

\begin{frame}{Conectare la distanță}
  \begin{itemize}
    \item ssh ana@rosedu.org
    \item ssh -l bogdan rosedu.org
    \item ssh -p 2222 -l corina rosedu.org
    \item ssh -l dan rosedu.org "ls \~{}/projects"
    \item ssh -i .ssh/servers.priv -l root rosedu.org
    \item man ssh
  \end{itemize}
\end{frame}

\begin{frame}{Copiere la/de la distanță}
  \begin{itemize}
    \item scp file.txt elena@rosedu.org:
    \item scp file.txt florin@rosedu.org:projects/elena/tmp/
    \item scp -P 2222 file.txt gabi@rosedu.org:
    \item scp horia@rosedu.org:test.txt .
    \item scp -r horia@rosedu.org:docs/ \~{}/stuff/
    \item man scp
  \end{itemize}
\end{frame}

\begin{frame}{Generare pereche de chei}
  \begin{itemize}
    \item ssh-keygen -t rsa
    \item ssh-keygen -t dsa -N My$|$P422)(
    \item implicit \{\~{}/.ssh/id\_dsa, \~{}/.ssh/id\_dsa.pub\}
    \item ssh-keygen -t rsa -f my\_key
    \item \{my\_key, my\_key.pub\}
	\item ssh-keygen -H
    \item man ssh-keygen
  \end{itemize}
\end{frame}

\begin{frame}{Autentificare folosind chei publice}
  \begin{itemize}
        \item scp \~{}/.ssh/id\_rsa.pub irina@rosedu.org
        \item cat id\_rsa.pub $>>$ \~{}/.ssh/authorized\_keys
	\end{itemize}
	\pause \begin{itemize}
    	\item cat \~{}/.ssh/id\_rsa.pub $|$ ssh irina@rosedu.org "cat
$>>$ \~{}/.ssh/authorized\_keys"
	\end{itemize}
	\pause \begin{itemize}
		\item ssh-copy-id -i ~/.ssh/id\_rsa.pub irina@rosedu.org
  \end{itemize}
\end{frame}

\section{No\c{t}iuni avansate}
\frame{\tableofcontents[currentsection]}

\begin{frame}{Instalare și interacțiune server SSH}
  \begin{itemize}
  	\item emerge openssh
    \item apt-get install openssh-server
    \item /etc/init.d/ssh \{stop, start, restart, \pause reload\}
  \end{itemize}
\end{frame}

\begin{frame}{Configurări de bază server SSH}
  \begin{itemize}
    \item /etc/ssh/sshd\_config
    \item Port 22
    \item PasswordAuthentication no
    \item X11Forwarding yes
	\item GatewayPorts yes
    \item man sshd\_config
  \end{itemize}
\end{frame}

\begin{frame}{Folosire ssh-agent}
  \begin{itemize}
    \item mai multe perechi de chei $\rightarrow$ folosire opțiune -i la ssh/scp
    \item chei cu passphrase $\rightarrow$ introducerea passphrase-ului la fiecare conectare
    \item booooring (lene, pierdere de timp, rutină)
	\item ssh-agent bash
    \item ssh-add    ; se adaugă cheia implicită
    \item ssh-add \~{}/.ssh/my\_key    ; se adaugă cheia specificată
    \item la autentificare nu se mai cere passphrase; nu mai e nevoie de -i
    \item pornit automat cu interfața grafică
  \end{itemize}
\end{frame}

\begin{frame}{TCP Port Forwarding}
  \begin{itemize}
    \item ssh -L localhost:8080:rosedu.org:80 rtt@rosedu.org
      \begin{itemize}
        \item conexiune pe localhost:8080 sunt dirijate către
rosedu.org:80
      \end{itemize}
    \item ssh -R rosedu.org:8022:localhost:22 rtt@rosedu.org 
       \begin{itemize}
         \item conexiunile ajung la sistemul care a inițiat reverse port
		 forward
       \end{itemize}
  \end{itemize}
\end{frame}

\begin{frame}{SOCKS proxy}
  \begin{itemize}
  	\item ssh -D 8080 rtt@rosedu.org
	\item asigură un canal criptat până la server
	\item proxy anonim
	\item tunel local, dinamic
  \end{itemize}
\end{frame}

\begin{frame}{X-Forwarding cu SSH}
  \begin{itemize}
    \item X11Fowarding on ; /etc/ssh/sshd\_config
    \item ssh -X -l lucian rosedu.org
    \item xeyes, firefox, gvim etc.
    \item rulare la distanță, afișare locală
  \end{itemize}
\end{frame}

\section{Utilizare SSH \^{i}n alte aplicații}
\frame{\tableofcontents[currentsection]}

\begin{frame}{SCM}
  \begin{itemize}
    \item git clone ssh://rosedu.org/projects/repo.git/
    \item svn checkout svn+ssh//rosedu.org/projects/repo.svn/
  \end{itemize}
\end{frame}

\begin{frame}{rsync}
  \begin{itemize}
    \item rsync -avz -e ssh monica@rosedu.org:/home/cdl/projects
\~{}/rsync/
  \end{itemize}
\end{frame}

\begin{frame}{SSHFS}
	\begin{itemize}
	\item sistem de fişiere peste ssh
	\item implementat cu FUSE (Filesystem in Userspace)
	\end{itemize}
\end{frame}

\section{Exemple}
\frame{\tableofcontents[currentsection]}
\begin{frame}{Exemple}
	\begin{itemize}
	\item tunel invers creat deja de pe o sație din regie (sper să fi ținut) 
	\item ssh -Nn -R rosedu.org:29000:localhost:22 rtt@rosedu.org
	\item GatewayPorts
	\item tunel direct, tunel invers (între rosedu.org, şi calculatorul de
	acasă)
	\item proxy socks5	(tsock wget whatismyip.org)
	\item montare sistem de fişiere (sshfs)
	\item lansare aplicație X
	\end{itemize}
\end{frame}

\begin{frame}{Bibliografie}
  \begin{itemize}
    \item SSH: The Secure Shell (The Definitive Guide), O'Reilly 2005 (2nd
Edition)
    \item \url{http://www.linuxjournal.com/article/4412}
    \item \url{http://suso.org/docs/shell/ssh.sdf}
edition)
  \end{itemize}
\end{frame}

\end{document}
