% vim: set tw=78 aw:
\documentclass{beamer}

\usepackage[utf8x]{inputenc}		% diacritice
\usepackage[romanian]{babel}
\usepackage{color}			% highlight
\usepackage{alltt}			% highlight
\usepackage{code/highlight}		% highlight
\usepackage{hyperref}			% folosiți \url{http://...}
					% sau \href{http://...}{Nume Link}
\mode<presentation>
{ \usetheme{Berlin} }			% TODO: settle this

% Încărcăm simbolurilor Unicode românești în titlu și primele pagini
\PreloadUnicodePage{200}

\title[Preprocesorul C]{Preprocesorul C}
\subtitle{Tech Talks}
\institute{ROSEdu}
\author[Răzvan]{Răzvan Deaconescu \\ \texttt{razvan@rosedu.org}}
\date{10 septembrie 2009}

\begin{document}

% Slide-urile cu mai multe părți sunt marcate cu textul (cont.)
\setbeamertemplate{frametitle continuation}[from second]

% Arătăm numărul frame-ului
\setbeamertemplate{footline}[frame number]

\frame{\titlepage}

\frame{\tableofcontents}

% NB: Secțiunile nu sunt marcate vizual, ci doar apar în cuprins
\section{No\textcommabelow{t}iuni generale}

% Pentru reamintirea periodică a cuprinsului și unde ne aflăm:
\frame{\tableofcontents[currentsection]}

% Titlul unui frame se specifică fie în acolade, imediat după \begin{frame},
% fie folosind \frametitle
\begin{frame}{Rolul preprocesorului C}
  \begin{itemize}		% Just like normal LaTeX
    \item Includerea fișierelor header
    \item Expandarea macro-urilor
    \item Compilare condiționată
    \item Diagnosticare
  \end{itemize}
\end{frame}

\begin{frame}[allowframebreaks] % Spargem paginile mai mari automat
  % Atenție: allowframebreaks nu funcționează cu overlay-uri
  \frametitle{Invocarea preprocesorului}
  \begin{itemize}
    \item bla
  \end{itemize}
\end{frame}

\begin{frame}{Exemple de utilizare}
  \begin{itemize}
    \item bla
  \end{itemize}
\end{frame}

\section{Macrodefiniții}

\frame{\tableofcontents[currentsection]}

\begin{frame}{Directivele \#define și \#undef}
  \begin{itemize}
    \item bla
  \end{itemize}
\end{frame}

\begin{frame}{Macro-uri simple (object-like macros)}
  \begin{itemize}
    \item bla
  \end{itemize}
\end{frame}

\begin{frame}{Folosirea parantezelor}
  \begin{itemize}
    \item bla
  \end{itemize}
\end{frame}

\begin{frame}{Macro-uri vs. const}
  \begin{itemize}
    \item bla
  \end{itemize}
\end{frame}

\begin{frame}{Macro-uri vs. enum}
  \begin{itemize}
    \item bla
  \end{itemize}
\end{frame}

\begin{frame}{Macro-uri predefinite}
  \begin{itemize}
    \item bla
    \item gcc -dM -E
  \end{itemize}
\end{frame}

\begin{frame}{Definirea de macro-uri în linia de comandă}
  \begin{itemize}
    \item bla
  \end{itemize}
\end{frame}

\begin{frame}{Macro-uri cu parametri (function-like macros)}
  \begin{itemize}
    \item bla
  \end{itemize}
\end{frame}

\begin{frame}{Macro-uri vs. funcții inline}
  \begin{itemize}
    \item bla
  \end{itemize}
\end{frame}

\begin{frame}{Macro-uri cu număr variabil de argumente (variadic macros)}
  \begin{itemize}
    \item bla
  \end{itemize}
\end{frame}

\begin{frame}{Efecte laterale}
  \begin{itemize}
    \item bla
  \end{itemize}
\end{frame}

\section{Preprocesare condiționată}

\frame{\tableofcontents[currentsection]}

\begin{frame}{Directive de preprocesare condiționată}
  \begin{itemize}
    \item bla
  \end{itemize}
\end{frame}

\begin{frame}{Asigurarea portabilității}
  \begin{itemize}
    \item bla
  \end{itemize}
\end{frame}

\begin{frame}{"Eliminarea" codului}
  \begin{itemize}
    \item bla
  \end{itemize}
\end{frame}

\begin{frame}{Activare/dezactivare mod debug}
  \begin{itemize}
    \item bla (mai mult text aici)
  \end{itemize}
\end{frame}

\begin{frame}{Combinare cod C și cod C++}
  \begin{itemize}
    \item bla (mai mult text aici)
  \end{itemize}
\end{frame}

\section{Includerea fișierelor header}

\frame{\tableofcontents[currentsection]}

\begin{frame}
  \frametitle{Directiva \#include}
  \begin{itemize}
    \item bla
  \end{itemize}
\end{frame}

\begin{frame}{Exemplu de structură de fișier header}
  \begin{itemize}
    \item macro-uri de gardă/control
  \end{itemize}
\end{frame}

\begin{frame}{Ce conține un fișier header?}
  \begin{itemize}
    \item bla
  \end{itemize}
\end{frame}

\begin{frame}{De ce nu se include un fișier C?}
  \begin{itemize}
    \item bla
  \end{itemize}
\end{frame}

\begin{frame}{\#include "file.h" vs. \#include $<$file.h$>$}
  \begin{itemize}
    \item bla
  \end{itemize}
\end{frame}

\begin{frame}{Extinderea căii de căutare a fișierelor header}
  \begin{itemize}
    \item bla
  \end{itemize}
\end{frame}

\section{Altele}

\frame{\tableofcontents[currentsection]}

\begin{frame}{Stringification}
  \begin{itemize}
    \item bla
    \item valoarea unei variabile DBG(a)
  \end{itemize}
\end{frame}

\begin{frame}{Concatenare}
  \begin{itemize}
    \item bla
  \end{itemize}
\end{frame}

\begin{frame}{Diagnosticare}
  \begin{itemize}
    \item bla (\#error)
    \item bla (\#warning)
  \end{itemize}
\end{frame}

\section{Resurse utile}

\frame{\tableofcontents[currentsection]}

\begin{frame}{Link-uri}
  \begin{itemize}
    \item \url{http://en.wikipedia.org/wiki/C\_preprocessor}
    \item \url{http://c-faq.com/cpp/index.html}
  \end{itemize}
\end{frame}

\begin{frame}{Altele}
  \begin{itemize}
    \item man cpp
    \item info cpp
    \item comp.lang.c (Usenet)
    \item \#\#c (IRC, Freenode)
  \end{itemize}
\end{frame}

\begin{frame}{Întrebări}
\end{frame}

\end{document}
