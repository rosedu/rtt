% vim: set tw=78 tabstop=4 shiftwidth=4 aw ai:
\documentclass{beamer}

\usepackage[utf8x]{inputenc}		% diacritice
\usepackage[romanian]{babel}
\usepackage{color}			% highlight
\usepackage{alltt}			% highlight

% highlight; comment this out in case you don't input code source files
%\usepackage{code/highlight}		% highlight

\usepackage{hyperref}			% folosiți \url{http://...}
					% sau \href{http://...}{Nume Link}
\usepackage{verbatim}

% Încărcăm simbolurilor Unicode românești în titlu și primele pagini
\PreloadUnicodePage{200}

\mode<presentation>
{ \usetheme{Berlin} }

\title[CLI Eficient]{Utilizarea eficientă a liniei de comandă}
\subtitle{From GUI-zero to CLI-hero}
\institute{ROSEdu Tech Talks}
\author[Răzvan Deaconescu]{Răzvan Deaconescu\\
	razvan@rosedu.org}
\date{9 octombrie 2010}

\begin{document}

% Slide-urile cu mai multe părți sunt marcate cu textul (cont.)
\setbeamertemplate{frametitle continuation}[from second]

% Arătăm numărul frame-ului
%\setbeamertemplate{footline}[frame number]

% Show contents at every section beginning. Ripped off from manual.
\AtBeginSection[] % Do nothing for \section*
{
	\begin{frame}<beamer>
		\frametitle{Cuprins}
	\tableofcontents[currentsection]
		\end{frame}
}

\frame{\titlepage}

\frame{\tableofcontents}

\begin{frame}{Moto}
	\textit{It was a mistake to think that GUIs ever would, could, or even
	   should, eliminate CLIs.}\\
	\vspace{5mm}
	\hfill \textit{Jeffrey Snover (Architect of Windows PowerShell)}
\end{frame}

\begin{frame}{Despre această prezentare}
	\begin{itemize}
		\item Dacă înțelegeți
			\begin{itemize}
				\item 0\% -- puneți întrebări
				\item 5\% -- e bine
				\item 25\% -- e excelent
				\item 50\% -- e ceva suspect la mijloc
				\item 75\% -- suntem în altă dimensiune
			\end{itemize}
		\item Scopul: uite ce se poate face!
		\item Reacții așteptate
			\begin{itemize}
				\item Huh?
				\item I've a feeling we're not in Kansas any more.
				\item How'd he do that?
				\item Când o să fiu mare \ldots
				\item Arunc mouse-ul!
				\item Bine că mai știu niște matematică. Dacă nu merge cu
				calculatoarele astea \ldots
			\end{itemize}
	\end{itemize}
\end{frame}

% NB: Secțiunile nu sunt marcate vizual, ci doar apar în cuprins
\section{Linia de comandă}

% Titlul unui frame se specifică fie în acolade, imediat după \begin{frame},
% fie folosind \frametitle
\begin{frame}{Ce este linia de comandă?}
	\begin{itemize}		% Just like normal LaTeX
		\item interfață simplă, bazată pe text, de interacțiune cu o aplicație
		\item prompt -- text care definește zona în care se introduce text
		\item comandă -- zonă de text editabilă de utilizator care poate
		conduce la rularea unei funcționalități
		\item rularea unei comenzi -- se apasă, de obicei, ENTER și se execută
		comanda în cauză
		\item argumente -- opțiuni ale unei comenzi pentru a defini facilități
		suplimentare
		\item CLI -- comand line interface
	\end{itemize}
\end{frame}

\begin{frame}{Exemple de CLI}
	\begin{itemize}
		\item shell + terminal Unix
		\item Command Prompt / Power Shell
		\item echipamente de rețea
		\item configurarea jocurilor (în special FPS-uri)
		\item IRC (/away, /msg, /help)
		\item MATLAB
		\item AutoCAD
		\item clienți de aplicații de baze de date
		\item Python
	\end{itemize}
\end{frame}

\begin{frame}{CLI, shell, terminal}
	\begin{itemize}
		\item CLI
			\begin{itemize}
				\item tip de interfață
				\item specific mai multor aplicații
			\end{itemize}
		\item shell
			\begin{itemize}
				\item aplicație care furnizează interfață de acces la serviciile sistemului de operare
				\item shell pentru alte aplicații
				\item CLI sau GUI
				\item pentru CLI -- interpretor de comenzi
			\end{itemize}
		\item terminal
			\begin{itemize}
				\item dispozitiv hardware pentru input/output
				\item system console
				\item virtual terminal (ALT+CTRL+F$[$1-6$]$)
				\item terminal emulator (gnome-terminal, konsole, PuTTY)
			\end{itemize}
	\end{itemize}
\end{frame}

\begin{frame}{Exemplu în Linux}
	\begin{itemize}
		\item \texttt{razvan@valhalla:$\sim$\$ ls -l -h /tmp}
		\item \texttt{razvan@valhalla:$\sim$\$} -- prompt-ul
		\item \texttt{ls} -- comanda
		\item \texttt{-l -h /tmp} -- argumentele comenzii
	\end{itemize}
\end{frame}

\section{CLI vs. GUI}

\begin{frame}{GUI}
	\begin{itemize}
		\item WIMP -- Window, Image, Menu, Pointing Device
		\item eye candy
		\item personalizabilă
		\item mouse
	\end{itemize}
\end{frame}

\begin{frame}{CLI vs. GUI}
	\begin{columns}
		\begin{column}[l]{0.5\textwidth}
			\begin{itemize}
				\item ușurință în utilizare
				\item \textbf{control}
				\item multitasking
				\item \textbf{viteză}
				\item \textbf{resurse consumate}
				\item \textbf{automatizare}
				\item \textbf{acces la distanță}
			\end{itemize}
		\end{column}
		\begin{column}[l]{0.5\textwidth}
			\begin{itemize}
				\item \textbf{ușurință în utilizare}
				\item control
				\item \textbf{multitasking}
				\item viteză
				\item resurse consumate
				\item automatizare
				\item acces la distanță
			\end{itemize}
		\end{column}
	\end{columns}
\end{frame}

\begin{frame}{De ce CLI?}
	\begin{itemize}
		\item eficiență
		\item îmbunătățire continuă
		\item interfață unică
		\item flexibilitate
		\item Feel da powa!
		\item exemple
			\begin{itemize}
				\item parcurgerea directoarelor
				\item mutarea fișierelor
			\end{itemize}
	\end{itemize}
\end{frame}

\begin{frame}{De ce GUI?}
	\begin{itemize}
		\item grandma-friendly
		\item lucrul cu sistemul de fișiere
			\begin{itemize}
				\item redenumirea unor fișiere
				\item selectarea a N fișiere
			\end{itemize}
		\item trecere prin mai multe directoare
	\end{itemize}
\end{frame}

\section{Facilități shell}

\begin{frame}{Autocompletion}
	\begin{itemize}
		\item bash-completion
		\item TAB
			\begin{itemize}
				\item completarea unei comenzi
				\item completarea opțiunilor
			\end{itemize}
		\item TAB TAB
			\begin{itemize}
				\item furnizarea tuturor soluțiilor posibile
			\end{itemize}
	\end{itemize}
\end{frame}

\begin{frame}{Istoricul comenzilor}
	\begin{itemize}
		\item tastă sus, tastă jos
		\item !! -- rularea comenzii anterioare
		\item \$\_ sau ALT+. -- ultimul argument al ultimei comenzi
		\item CTRL+R -- reverse search
	\end{itemize}
\end{frame}

\begin{frame}{cd}
	\begin{itemize}
		\item fără argumente -- schimbarea directorului rădăcină
		\item \texttt{cd \$\_}, \texttt{cd ALT+.} (util după mkdir)
		\item \texttt{cd -} -- directorul anterior
	\end{itemize}
\end{frame}

\begin{frame}{Comment out}
	\begin{itemize}
		\item folosirea \# în fața unei comenzi o comentează
		\item puteți reveni ulterior în istoric
	\end{itemize}
\end{frame}

\begin{frame}{Shell expansion}
	\begin{itemize}
		\item $\sim$ -- expandare la home folder
		\item $\sim$username -- expandare la home folder-ul utilizatorului
		username
		\item variabile shell
	\end{itemize}
\end{frame}

\begin{frame}{Globbing}
	\begin{itemize}
		\item * -- orice caracter de oricâte ori
		\item \{a,b,c\} -- selecție între opțiuni
		\item $[$a-g$]$ -- clasă de caractere
		\item ? -- orice o singură dată
	\end{itemize}
\end{frame}

\begin{frame}{CDPATH}
	\begin{itemize}
		\item controlează directoarele către care se poate folosi cd și cale
		relativă
		\item \texttt{CDPATH=.} -- cd doar către directoarele vizibile din directorul
		curent
		\item \texttt{CDPATH=.:/home/razvan/projects} -- în plus, directoarele vizibile
		din projects
	\end{itemize}
\end{frame}

\begin{frame}{Redirectarea comenzilor}
	\begin{itemize}
		\item \texttt{command $>$ file} -- redirectarea ieșirii standard
		\item \texttt{command $<$ file} -- redirectarea intrării standard
		\item \texttt{command 2$>$ file} -- redirectarea ieșirii de eroare standard
		\item \texttt{cat /dev/null $>$ file} -- trunchierea fișierului
		\item \texttt{$>$ file} -- trunchierea fișierului
	\end{itemize}
\end{frame}

\begin{frame}{Înlănțuirea comenzilor}
	\begin{itemize}
		\item command1 $|$ command2 -- ieșirea comenzii command1 este intrare
		pentru command2
		\item filozofia Unix: Do one thing, do one thing well!
	\end{itemize}
\end{frame}

\begin{frame}{Shell scripting}
	\begin{itemize}
		\item variabile
		\item filtre de text
		\item cicluri, instrucțiuni de decizie
		\item automatizare
	\end{itemize}
\end{frame}

\section{Facilități libreadline}

\begin{frame}{libreadline}
	\begin{itemize}
		\item biblioteca folosită pentru editarea comenzilor
		\item folosită de multe aplicații CLI
	\end{itemize}
\end{frame}

\begin{frame}{Generalități}
	\begin{itemize}
		\item CTRL+D -- logout
		\item CTRL+C -- anularea comenzii (\textbf{foarte} important)
		\item CTRL+L -- clear screen
	\end{itemize}
\end{frame}

\begin{frame}{Mișcări}
	\begin{itemize}
		\item Emacs bindings
		\item CTRL+F -- caracter înainte
		\item CTRL+B -- caracter înapoi
		\item ALT+F -- cuvânt înainte
		\item ALT+B -- cuvânt înapoi
		\item CTRL+A -- început de rând
		\item CTRL+E -- sfârșit de rând
		\item CTRL+$]$+caracter -- căutarea primei apariției a caracterului
	\end{itemize}
\end{frame}

\begin{frame}{Editare}
	\begin{itemize}
		\item CTRL+D -- șterge caracterul curent (Delete)
		\item Backspace -- șterge caracterul anterior
		\item ALT+D -- șterge cuvântul curent
		\item ALT+Backspace -- șterge cuvântul anterior
		\item CTRL+K -- șterge din poziția curentă până la sfârșit
		\item CTRL+U -- șterge din poziția curentâ până la început
		\item CTRL+\_ -- undo
	\end{itemize}
\end{frame}

\begin{frame}{inputrc}
	\begin{itemize}
		\item fișierul de configurare pentru libreadline
		\item something cute -- ALT+\#
		\item \texttt{set completion-ignore-case on}
		\item mai multe în documentație (\texttt{man bash})
	\end{itemize}
\end{frame}

\section{Extra juice}

\begin{frame}{Acces rapid la shell}
	\begin{itemize}
		\item keyboard shortcut pentru terminal -- acces rapid
		\item nautilus-open-terminal
			\begin{itemize}
				\item pachet GNOME
				\item deschide un terminal în directorul din interfața grafică
			\end{itemize}
	\end{itemize}
\end{frame}

\begin{frame}{Rulare rapidă de aplicații}
	\begin{itemize}
		\item gnome-open
			\begin{itemize}
				\item pachet GNOME
				\item deschide fișierul transmis ca argument cu aplicația
				dorită
				\item un fel de ``CLI double click''
			\end{itemize}
		\item alias-uri
			\begin{itemize}
				\item pentru comenzi comune/frevente
				\item \texttt{alias go='gnome-open'}
				\item \texttt{alias sshsr='ssh -l root swarm.cs.pub.ro'}
				\item \texttt{alias grep='grep --color=auto'}
			\end{itemize}
	\end{itemize}
\end{frame}

\begin{frame}{Scurtături în mediul grafic}
	\begin{itemize}
		\item ALT+F2 -- rulare comandă
		\item ALT+F4 -- închidere fereastră
		\item ALT+F5 -- minimizare
		\item ALT+F10 -- maximizare
		\item ALT+F8 -- redimensionare
		\item ALT+F7 -- mutare
	\end{itemize}
\end{frame}

\section{Concluzii}

\begin{frame}{Cuvinte cheie}
	\begin{columns}
		\begin{column}[l]{0.5\textwidth}
			\begin{itemize}
				\item CLI
				\item shell
				\item terminal
				\item prompt
				\item comandă
				\item argumente
				\item GUI
				\item CLI
				\item autocompletion
				\item TAB, TAB-TAB
			\end{itemize}
		\end{column}
		\begin{column}[l]{0.5\textwidth}
			\begin{itemize}
				\item history
				\item reverse search
				\item globbing
				\item scripting
				\item libreadline
				\item editarea comenzilor
				\item inputrc
				\item nautilus-open-terminal
				\item gnome-open
				\item alias-uri
			\end{itemize}
		\end{column}
	\end{columns}
\end{frame}

\begin{frame}{Resurse utile}
	\begin{itemize}
		\small
		\item \url{http://en.wikipedia.org/wiki/Command-line\_interface}
		\item
		\url{http://www.gnu.org/software/bash/manual/bashref.html\#Command-Line-Editing}
		\item
		\url{http://www.gnu.org/software/bash/manual/bashref.html\#Using-History-Interactively}
		\item \url{http://www.computerhope.com/issues/ch000619.htm}
		\normalsize
		\item ``command line interface advantages'' pe Google
		\item Andrew Hunt and David Thomas -- The Pragmatic Programmer
		(Chapter 3. Basic Tools)
	\end{itemize}
\end{frame}

\section{Întrebări}

\end{document}
