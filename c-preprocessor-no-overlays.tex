% vim: set tw=78 tabstop=4 shiftwidth=4 aw:
\documentclass{beamer}

\usepackage[utf8x]{inputenc}		% diacritice
\usepackage[romanian]{babel}
\usepackage{color}			% highlight
\usepackage{alltt}			% highlight
\usepackage{code/highlight}		% highlight
\usepackage{hyperref}			% folosiți \url{http://...}
					% sau \href{http://...}{Nume Link}
\mode<presentation>
{ \usetheme{Berlin} }			% TODO: settle this

% Încărcăm simbolurilor Unicode românești în titlu și primele pagini
\PreloadUnicodePage{200}

\title[Preprocesorul C]{Preprocesorul C}
\subtitle{Tech Talks}
\institute{ROSEdu}
\author[Răzvan]{Răzvan Deaconescu \\ \texttt{razvan@rosedu.org}}
\date{10 octombrie 2009}

\begin{document}

% Slide-urile cu mai multe părți sunt marcate cu textul (cont.)
\setbeamertemplate{frametitle continuation}[from second]

% Arătăm numărul frame-ului
\setbeamertemplate{footline}[frame number]

\frame{\titlepage}

\frame{\tableofcontents}

% NB: Secțiunile nu sunt marcate vizual, ci doar apar în cuprins
\section{No\textcommabelow{t}iuni generale}

% Pentru reamintirea periodică a cuprinsului și unde ne aflăm:
\frame{\tableofcontents[currentsection]}

% Titlul unui frame se specifică fie în acolade, imediat după \begin{frame},
% fie folosind \frametitle
\begin{frame}{Rolul preprocesorului C}
	\begin{itemize}		% Just like normal LaTeX
		\item Includerea fișierelor header
		\item Expandarea macro-urilor
		\item Compilare condiționată
		\item Diagnosticare
	\end{itemize}
\end{frame}

\begin{frame}{Invocarea preprocesorului}
	\begin{itemize}
		\item Intern de compilator
			\begin{beamerboxesrounded}[lower=block body,shadow=true]{}
				\large{\texttt{gcc -Wall file.c}}
			\end{beamerboxesrounded}
		\item Intern de compilator doar pentru faza preprocesării
			\begin{beamerboxesrounded}[lower=block body,shadow=true]{}
				\large{\texttt{gcc -Wall -E file.c -o file.i}}
			\end{beamerboxesrounded}
		\item Cu ajutorul comenzii \texttt{cpp}
			\begin{beamerboxesrounded}[lower=block body,shadow=true]{}
				\large{\texttt{cpp file.c file.i}}
			\end{beamerboxesrounded}
	\end{itemize}
\end{frame}

\begin{frame}{Exemplu de utilizare}
	\begin{columns}
		\begin{column}[l]{0.5\textwidth}
			\input{code/sample}
		\end{column}
		\begin{column}[l]{0.4\textwidth}
			\input{code/sample-preproc}
		\end{column}
	\end{columns}
\end{frame}

\section{Macrodefiniții}

\frame{\tableofcontents[currentsection]}

\begin{frame}{Directivele \#define și \#undef}
	\begin{itemize}
		\item \textbf{Directivele de preprocesare încep cu \textbf{\#}}
		\item Definirea și anularea definirii unui macro
		\item \texttt{\#define MY\_VAR  50}
		\item \texttt{\#undef MY\_VAR}
		\item \texttt{a = a + MY\_VAR} $\rightarrow$ \texttt{a = a + 50}
	\end{itemize}
\end{frame}

\begin{frame}{Macro-uri simple (object-like macros)}
	\begin{itemize}
		\item \texttt{\#define NUME	VALOARE}
		\item \texttt{NUME} respectă regulile de nume pentru o variabilă
		\item \texttt{VALOARE} poate să conțină aproape orice
		\item Se înlocuiește \texttt{NUME} cu \texttt{VALOARE} peste tot
	\end{itemize}
\end{frame}

\begin{frame}{Stupid example}
	\begin{columns}
		\begin{column}[l]{0.4\textwidth}
			\input{code/stupid-example}
		\end{column}
		\begin{column}[l]{0.4\textwidth}
			\input{code/stupid-example-preproc}
		\end{column}
	\end{columns}
\end{frame}

\begin{frame}{Folosirea parantezelor}
	\begin{itemize}
		\item O greșeală frecventă
		\item Once upon a time during a PSO lab (2006-2007)
			\begin{beamerboxesrounded}[lower=block body,shadow=true]{}
				\texttt{\#define NR\_syscalls     MY\_SYSCALL\_NO+1 \\
			... \\
				new\_syscall\_table = \\
\hlstd{}\hlstd{\ \ \ \ \ \ \ \ }\hlstd{}malloc(NR\_syscalls * sizeof(void *));
		}
			\end{beamerboxesrounded}
		\item \textbf{Nu} faceți așa
			\begin{beamerboxesrounded}[lower=block body,shadow=true]{}
				\texttt{\#define MAX(a, b)   (a > b ? a : b)}
			\end{beamerboxesrounded}
		\item Faceți așa
			\begin{beamerboxesrounded}[lower=block body,shadow=true]{}
				\texttt{\#define MAX(a, b)   ((a) > (b) ? (a) : (b))}
			\end{beamerboxesrounded}
		\item E OK tot timpul?
	\end{itemize}
\end{frame}

\begin{frame}{Macro-uri vs. const}
	\begin{itemize}
		\item Avantaje macro-uri
			\begin{itemize}
				\item mai rapide (faza de preprocesare)
				\item interpretate la preprocesare
				\item nu ocupă spațiu în memorie
			\end{itemize}
		\item Avantaje const
			\begin{itemize}
				\item type safety
				\item sunt, de fapt, variabile (au o adresă)
				\item în C++ se pot folosi pentru alocări statice
			\end{itemize}
	\end{itemize}
\end{frame}

\begin{frame}{Macro-uri vs. enum}
	\begin{itemize}
		\item Avantaje macro-uri
			\begin{itemize}
				\item flexibile (nu sunt limitate la valori întregi)
				\item nu impun constrângeri de spațiu ocupat
			\end{itemize}
		\item Avantaje enum
			\begin{itemize}
				\item type safety
				\item block scope
			\end{itemize}
	\end{itemize}
\end{frame}

\begin{frame}{Macro-uri predefinite}
	\begin{beamerboxesrounded}[lower=block body,shadow=true]{}
\texttt{\$ gcc -dM -E sample.c \\
\#define \_\_CHAR\_BIT\_\_ 8 \\
\#define \_\_unix\_\_ 1 \\
\#define \_\_x86\_64 1 \\
\#define \_\_SHRT\_MAX\_\_ 32767 \\
\#define \_\_linux 1 \\
\#define \_\_gnu\_linux\_\_ 1 \\
\#define SAMPLE\_H\_ 1 \\
\#define TEST\_NUM 25 \\
\#define MAX(a,b) ((a) > (b) ? (a) : (b)) \\
\#define \_\_STDC\_\_ 1
		}
	\end{beamerboxesrounded}
	\begin{beamerboxesrounded}[lower=block body,shadow=true]{}
\texttt{\$ gcc -dM -E sample.c | wc -l \\
131
}
	\end{beamerboxesrounded}
\end{frame}

\begin{frame}{Definirea de macro-uri în linia de comandă}
	\begin{beamerboxesrounded}[lower=block body,shadow=true]{}
		 \texttt{\$ gcc -DMY\_MACRO=40 -Wall comm.c \\
\$ ./a.out \\
MY\_MACRO = 40
		}
	\end{beamerboxesrounded}
	\begin{beamerboxesrounded}[lower=block body,shadow=true]{}
		\texttt{\$ gcc -dM -E code/sample.c | grep \_\_linux\_\_ \\
\#define \_\_linux\_\_ 1 \\
\$ gcc -U\_\_linux\_\_ -dM -E code/sample.c | grep \_\_linux\_\_ \\
\$
		}
	\end{beamerboxesrounded}
\end{frame}

\begin{frame}{Macro-uri cu parametri (function-like macros)}
			\begin{beamerboxesrounded}[lower=block body,shadow=true]{}
\texttt{\#define echo()  printf("Hello, World!\textbackslash{}n")}
			\end{beamerboxesrounded}
			\begin{beamerboxesrounded}[lower=block body,shadow=true]{}
\texttt{\#define max(a, b)   ((a) \textgreater{} (b) ? (a) : (b))}
			\end{beamerboxesrounded}
			\begin{beamerboxesrounded}[lower=block body,shadow=true]{}
\texttt{\#define do\_error(msg) \textbackslash{} \\
\hlstd{}\hlstd{\ \ \ \ \ \ \ \ }\hlstd{}perror(msg);	\textbackslash{} \\
\hlstd{}\hlstd{\ \ \ \ \ \ \ \ }\hlstd{}exit(EXIT\_FAILURE)
	}
			\end{beamerboxesrounded}
	\begin{itemize}
		\item E ceva greșit?
	\end{itemize}
			\begin{beamerboxesrounded}[lower=block body,shadow=true]{}
\texttt{if (error\_condition) do\_error();}
			\end{beamerboxesrounded}
\end{frame}

\begin{frame}{Macro-uri cu parametri (cont.)}
			\begin{beamerboxesrounded}[lower=block body,shadow=true]{}
\texttt{\#define do\_error(msg) \{ \textbackslash{} \\
\hlstd{}\hlstd{\ \ \ \ \ \ \ \ }\hlstd{}perror(msg);	\textbackslash{} \\
\hlstd{}\hlstd{\ \ \ \ \ \ \ \ }\hlstd{}exit(EXIT\_FAILURE); \textbackslash{} \\
\}
	}
			\end{beamerboxesrounded}
	\begin{itemize}
		\item Inconsistență - nu e nevoie de ; (punct și virgulă) la apel
	\end{itemize}
			\begin{beamerboxesrounded}[lower=block body,shadow=true]{}
\texttt{\#define do\_error(msg) do \{ \textbackslash{} \\
\hlstd{}\hlstd{\ \ \ \ \ \ \ \ }\hlstd{}perror(msg);	\textbackslash{} \\
\hlstd{}\hlstd{\ \ \ \ \ \ \ \ }\hlstd{}exit(EXIT\_FAILURE); \textbackslash{} \\
\} while (0)
	}
			\end{beamerboxesrounded}
	\begin{itemize}
		\item Perfect!
	\end{itemize}
\end{frame}

\begin{frame}{Macro-uri vs. funcții inline}
	\begin{itemize}
		\item Avantaje funcții inline
			\begin{itemize}
				\item type safety
				\item fără bătăi de cap
			\end{itemize}
		\item Avantaje macro-uri
			\begin{itemize}
				\item sunt portabile
				\item unele lucruri se realizează mai greu cu funcții inline
	(argumente transmise prin adresă)
				\item unele lucruri nu se pot face cu funcții inline
			\end{itemize}
	\end{itemize}

	\begin{beamerboxesrounded}[lower=block body,shadow=true]{}
		\texttt{\#define container\_of(ptr, type, member) \textbackslash{} \\
\hlstd{}\hlstd{\ \ \ \ }\hlstd{}(type *)( (char *)ptr -
offsetof(type, member)) \\
\#define offsetof(TYPE, MEMBER) \textbackslash{} \\
\hlstd{}\hlstd{\ \ \ \ }\hlstd{}((size\_t) \&((TYPE *)0)-\textgreater{}MEMBER)
		}
	\end{beamerboxesrounded}

\end{frame}

\begin{frame}{Macro-uri cu număr variabil de argumente (variadic macros)}
	\begin{itemize}
		\item Similar funcțiilor cu număr variabil de argumente
		\item În standardul C99
	\end{itemize}

	\begin{beamerboxesrounded}[lower=block body,shadow=true]{}
\texttt{\#define DEBUG(fmt, ...) \textbackslash{} \\
\hlstd{}\hlstd{\ \ \ \ \ \ \ \ }\hlstd{}fprintf(stderr, fmt, \_\_VA\_ARGS\_\_)}
	\end{beamerboxesrounded}

	\begin{itemize}
		\item O problemă: apel \texttt{DEBUG("problema");}
		\item Soluție: token paste operator (\textbf{\#\#}) (extensie GNU)
	\end{itemize}

	\begin{beamerboxesrounded}[lower=block body,shadow=true]{}
\texttt{\#define DEBUG(fmt, ...) \textbackslash{} \\
\hlstd{}\hlstd{\ \ \ \ \ \ \ \ }\hlstd{}fprintf(stderr, fmt, \#\#\_\_VA\_ARGS\_\_)}
	\end{beamerboxesrounded}
\end{frame}

\begin{frame}{Probleme în folosirea macro-urilor}
	\begin{itemize}
		\item Precedența operatorilor
			\begin{itemize}
				\item soluție: folosirea parantezelor
			\end{itemize}
		\item Omiterea operatorului ; (punct și virgulă)
			\begin{itemize}
				\item soluție: folosirea \texttt{do \{ ... \} while(0)}
			\end{itemize}
		\item Linii multiple
			\begin{itemize}
				\item soluție: folosirea operatorului \textbackslash{}
(backslash)
			\end{itemize}
	\end{itemize}
\end{frame}

\section{Preprocesare condiționată}

\frame{\tableofcontents[currentsection]}

\begin{frame}{Directive de preprocesare condiționată}
	\begin{itemize}
		\item \texttt{\#if expresie\_conditionala}
		\item \texttt{\#if defined macro} / \texttt{\#ifdef macro}
		\item \texttt{\#endif}
		\item \texttt{\#else}
		\item \texttt{\#elif}
	\end{itemize}
\end{frame}

\begin{frame}{Asigurarea portabilității}
	\input{code/win-lin}
\end{frame}

\begin{frame}{``Eliminarea'' codului}
	\begin{itemize}
		\item Util pentru ``comentarii''
		\item Trebuie să conțină cod ``preprocesabil''
	\end{itemize}

	\begin{beamerboxesrounded}[lower=block body,shadow=true]{}
\texttt{\#if 0 \\
\hlstd{}\hlstd{\ \ \ \ \ \ \ \ }\hlstd{}aaaa \\
\hlstd{}\hlstd{\ \ \ \ \ \ \ \ }\hlstd{}/*  \\
\hlstd{}\hlstd{\ \ \ \ \ \ \ \ }\hlstd{}... \\
\hlstd{}\hlstd{\ \ \ \ \ \ \ \ }\hlstd{}*/ \\
\hlstd{}\hlstd{\ \ \ \ \ \ \ \ }\hlstd{}if ...\\
\hlstd{}\hlstd{\ \ \ \ \ \ \ \ }\hlstd{}for ...\\
\hlstd{}\hlstd{\ \ \ \ \ \ \ \ }\hlstd{}... \\
\#endif
}
	\end{beamerboxesrounded}

\end{frame}

\begin{frame}{Activare/dezactivare mod debug}
	\input{code/true-debug}
	\begin{itemize}
		\item Se poate defini un macro în același fișier header
		\item Se poate folosi linia de comandă \texttt{gcc -DDEBUG\_\_ ...}
		\item Avansat, se pot stabili niveluri de jurnalizare
(\texttt{DEBUG\_LEVEL} + \texttt{Dprintf\_info}, \texttt{Dprintf\_warn} etc.)
	\end{itemize}
\end{frame}

\begin{frame}{Combinare cod C și cod C++}
	\begin{itemize}
		\item Compilatorul de C++ face ``name mangling''
		\item Linker-ul nu recunoaște simbolurile
	\end{itemize}
	\input{code/c-cpp}
\end{frame}

\section{Includerea fișierelor header}

\frame{\tableofcontents[currentsection]}

\begin{frame}
	\frametitle{Directiva \#include}
	\begin{itemize}
		\item Permite inserarea unui fișier header
		\item Se copiază conținutul acelui fișier
		\item Poate fi (și de obicei este) folosită pe mai multe niveluri (un
fișier header include alte fișiere header)
	\end{itemize}
\end{frame}

\begin{frame}{Exemplu de structură de fișier header}
	\begin{beamerboxesrounded}[lower=block body,shadow=true]{}
		\texttt{\#ifndef MY\_HEADER\_H\_ \\
\#define MY\_HEADER\_H\_	1 \\
... \\
\#endif \\
}
	\end{beamerboxesrounded}
	\begin{itemize}
		\item Macro-uri de gardă/control
		\item Previn includerea de mai multe ori a aceluiași fișier
		\item Denumite, în general \texttt{\_MY\_HEADER\_H},
\texttt{\_\_MY\_HEADER\_H} (rezervate pentru biblioteca standard),
\texttt{MY\_HEADER\_H\_}, \texttt{MY\_HEADER\_H\_\_}
	\end{itemize}
\end{frame}

\begin{frame}{Ce conține un fișier header?}
	\begin{itemize}
		\item Macro de gardă (\texttt{\#ifndef ...})
		\item Includerea altor fișiere header
			\begin{itemize}
				\item se folosesc tipuri definite în alt header
				\item se folosesc funcții declarate în alt header (vezi \texttt{Dprintf})
			\end{itemize}
		\item Macro-uri
		\item Structuri de date, enumerări și definiri de tipuri
		\item Declarații \textbf{externe} de variabile
		\item Declarații (antete) de funcții
		\item Definiții de funcții statice
		\item Definiții de funcții inline (de asemenea statice)
		\item Sfârșit macro de gardă (\texttt{\#endif})
	\end{itemize}
\end{frame}

\begin{frame}{De ce nu se include un fișier C?}
	\begin{itemize}
		\item Un fișier C conține funcții și variabile globale care pot fi non-statice
		\item Dacă acel fișier e inclus de două fișiere diferite care sunt
link-editate apare conflict de forma \textit{already defined}
		\item Un fișier header \textbf{nu} trebuie să aibă forma unui fișier C
			\begin{itemize}
				\item fără definiții de variabile globale non-statice
				\item fără definiții de funcții non-statice
			\end{itemize}
	\end{itemize}
\end{frame}

\begin{frame}{\#include \textless{}file.h\textgreater{} vs. \#include "file.h"}
	\begin{itemize}
		\item \texttt{\#include \textless{}file.h\textgreater{}}
			\begin{itemize}
				\item system headers
				\item căutare într-un set de directoare predefinite
				\item localizate preponderent în \texttt{/usr/include} și
\texttt{/usr/local/include}
			\end{itemize}
		\item \texttt{\#include "file.h"}
			\begin{itemize}
				\item local headers
				\item căutare începând cu directorul local
			\end{itemize}
	\end{itemize}
\end{frame}

\begin{frame}{Extinderea căii de căutare a fișierelor header}
	\begin{itemize}
		\item \texttt{CPPFLAGS = -I/usr/include/gdk/ -iquote../include}
		\item Forma \texttt{\#include \textless{}file.h\textgreater{}} caută
întâi în directoarele marcate cu \texttt{-I} și apoi în cele implicite
		\item Forma \texttt{\#include "file.h"} caută în directoarele marcate
cu \texttt{-iquote}, apoi în cele marcate cu \texttt{-I}, apoi în directorul
local și apoi în cele implicite
	\end{itemize}
\end{frame}

\section{Altele}

\frame{\tableofcontents[currentsection]}

\begin{frame}{Stringification}
	\begin{itemize}
		\item Operatorul \texttt{\#} în fața unui argument al unui macro
		\item Construcția este un șir de caractere reprezentând \textit{numele} argumentului
	\end{itemize}

	\input{code/string}
\end{frame}

\begin{frame}{Concatenare}
	\begin{itemize}
		\item Compilatorul concatenează șirurile adiacente
(\textit{string literal concatenation})
			\begin{itemize}
				\item \texttt{"an" "a" " a" "re"} $\rightarrow$
\texttt{"ana are"}
			\end{itemize}
		\item Operatorul \texttt{\#\#} concatenează doi tokeni într-unul
singur
		\item Wanna see something really scary?
	\end{itemize}

	\input{code/concat}
\end{frame}

\begin{frame}{Diagnosticare}
	\begin{itemize}
		\item Directiva \texttt{\#error} forțează eroare de preprocesare și
încheierea acțiunii
		\item Directiva \texttt{\#warning} permite continuarea procesării
	\end{itemize}

	\input{code/diag}
\end{frame}

\section{Resurse utile}

\frame{\tableofcontents[currentsection]}

\begin{frame}{Link-uri}
	\begin{itemize}
		\item \url{http://en.wikipedia.org/wiki/C\_preprocessor}
		\item \url{http://c-faq.com/cpp/index.html}
	\end{itemize}
\end{frame}

\begin{frame}{Altele}
	\begin{itemize}
		\item man cpp
		\item info cpp
		\item comp.lang.c (Usenet)
		\item \#\#c (IRC, Freenode)
	\end{itemize}
\end{frame}

\begin{frame}{Întrebări}
\end{frame}

\end{document}
